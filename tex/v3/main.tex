\documentclass{article}

% if you need to pass options to natbib, use, e.g.:
% \PassOptionsToPackage{numbers, compress}{natbib}
% before loading neurips_2019

% ready for submission
\usepackage{neurips_2019}

% to compile a preprint version, e.g., for submission to arXiv, add add the
% [preprint] option:
%\usepackage[preprint]{neurips_2019}

% to compile a camera-ready version, add the [final] option, e.g.:
% \usepackage[]{neurips_2019}

% to avoid loading the natbib package, add option nonatbib:
% \usepackage[nonatbib]{neurips_2019}

\usepackage[utf8]{inputenc} % allow utf-8 input
\usepackage[T1]{fontenc}    % use 8-bit T1 fonts
\usepackage{hyperref}       % hyperlinks
\usepackage{url}            % simple URL typesetting
\usepackage{booktabs}       % professional-quality tables
\usepackage{amsfonts}       % blackboard math symbols
\usepackage{nicefrac}       % compact symbols for 1/2, etc.
\usepackage{microtype}      % microtypography

%--------------------------------------------------------------------------------------------------------------------------------% 

\RequirePackage{mathrsfs, amsthm, amsmath, amsfonts, amssymb, mathtools}%
\usepackage[titletoc,title]{appendix}%
\usepackage{dsfont}%
\usepackage{caption} % 
\usepackage{subcaption}
\usepackage{float}%
\usepackage{graphicx, color}% 
\usepackage{bm}%bold equations

%% formatting appendices
\usepackage{etoolbox}
\usepackage{natbib}
\patchcmd{\appendices}{\quad}{. }{}{}

%% tables
\usepackage{pgfplots}%
\usepackage{booktabs,multirow,array,multicol}
\newcommand{\otoprule}{\midrule[\heavyrulewidth]}

%% algorithms
\usepackage[linesnumbered,ruled,vlined]{algorithm2e}
\usepackage{enumitem}

\setlist[itemize]{leftmargin=4.5mm}

\usepackage{accents}
\usepackage{datenumber}
\usepackage{wrapfig}

\newcommand{\rev}[1]{{\color{blue} #1}}
\newcommand{\ar}[1]{{\color{red} #1}}
\newcommand{\rf}[1]{{\color{green} #1}}

%% sections
\newtheorem{theorem}{Theorem}
\newtheorem{corollary}{Corollary}
\newtheorem{lemma}{Lemma}
\newtheorem{fact}{Fact}
\newtheorem{proposition}{Proposition}
\newtheorem{definition}{Definition}%[section]
\newtheorem*{notation}{Notation}%[section]
\newtheorem{discus}{Discussion}%[section]
\newtheorem{remark}{Remark}%[section]
\newtheorem{example}{Example}%[section]
\newtheorem{exs}{Examples}%[section]
\newtheorem{ca}{Cas}
\newtheorem{remarks}{Remarks}%[section]
\newtheorem{assumption}{Assumption}{\bf}{\rm}%

% math macros
\newcommand{\bigO}{\mathcal{O}}
\newcommand{\inr}[1]{\langle #1 \rangle}
\newcommand{\ind}[1]{{\mathds{1}}_{{#1}}}%
\newcommand{\norm}[1]{\|#1\|}
\newcommand{\R}{{\mathbb{R}}}
\newcommand{\E}{\mathds{E}}
\newcommand{\bcdot}{\raisebox{-0.80ex}{\scalebox{1.8}{$\cdot$}}}
\newcommand{\varsig}{\raisebox{-0.15ex}{\scalebox{1.30}{$\varsigma$}}}
\newcommand*{\dt}[1]{\accentset{\mbox{\large\bfseries .}}{#1}}

%% math operators
\DeclareMathOperator*{\argmin}{\arg\!\min}
\DeclareMathOperator*{\argmax}{\arg\!\max}
%--------------------------------------------------------------------------------------------------------------------------------%

\title{Screening Sinkhorn Algorithm for Regularized Optimal Transport}

\begin{document}

\author{%
Mokhtar Z. Alaya \\
LITIS EA4108\\
University of Rouen\\
\texttt{mokhtarzahdi.alaya@gmail.com} 
\And
Maxime Bérar\\
LITIS EA4108\\
University of Rouen\\
\texttt{maxime.berar@univ-rouen.fr} \\
\And
Gilles Gasso \\
LITIS EA4108\\
INSA, University of Rouen\\
\texttt{gilles.gasso@insa-rouen.fr} 
\And
Alain Rakotomamonjy\\
LITIS EA4108 \\
University of Rouen\\
and Criteo AI Lab, Criteo Paris \\
\texttt{alain.rakoto@insa-rouen.fr} \\
}

\maketitle

\begin{abstract}
We introduce in this paper a novel strategy for efficiently approximate the Sinkhorn distance between two discrete measures. After identifying neglectible components
of the dual solution of the regularized Sinkhorn problem, we propose to screen those components by directly setting them at that value before entering the Sinkhorn problem. This allows us to solve a smaller Sinkhorn problem while ensuring approximation with provable guarantees.
More formally, the approach is based on a reformulation of \emph{dual of Sinkhorn divergence problem} and on the KKT optimality conditions of this problem, which enable identification of dual components to be screened.
This new analysis leads to the \emph{Screenkhorn} algorithm.
We illustrate the efficiency of Screenkhorn on complex tasks such as dimensionality reduction or domain adaptation involving regularized optimal transport.
\end{abstract}

%!TEX root = main.tex

\section{Introduction} % (fold)
\label{sec:introduction}

Computing OT distances between pairs of probability measures or histograms, such as the earth mover's distance~\citep{werman1985,Rubner2000} and Monge-Kantorovich or Wasserstein distance~\citep{villani09optimal}, are currently generating an increasing attraction in different machine learning tasks~\citep{pmlr-v32-solomon14,kusnerb2015,pmlr-v70-arjovsky17a,ho2017}, statistics~\citep{frogner2015nips,panaretos2016,ebert2017ConstructionON,bigot2017,flamary2018WDA}, and computer vision~\citep{bonnel2011,Rubner2000,solomon2015}, among other applications~\citep{klouri17,peyre2019COTnowpublisher}.
In many of these problems, OT exploits the geometric features of the objects at hand in the underlying spaces to be leveraged in comparing probability measures.
This effectively leads to improve performance of methods that are oblivious to the geometry, for example the chi-squared distances or the Kullback-Leibler divergence.
Unfortunately, this advantage comes at the price of an enormous computational cost of solving the OT problem, that can be prohibitive in large scale applications.
For instance, the OT between two histograms with supports of equal size $n$ can be formulated as a linear programming problem that requires generally super $\bigO(n^3)$~\citep{pele2009} arithmetic operations, which is problematic when $n$ is larger than $10^3.$

A remedy to the heavy computation burden of OT lies in a prevalent approach referred as regularized OT~\citep{cuturinips13} and operates by adding an entropic regularization penalty to the original problem.  
Such a regularization guarantees a unique solution, since the objective function is strongly convex, and a greater computational stability.
Furthermore,~\citet{cuturinips13} proposed the so-called {dual of Sinkhorn divergence} as the dual of the entropic problem and noticed that finding the dual solution was equivalent to finding two diagonal matrices that made a full matrix bistochastic.
Therefore, the OT can be solved efficiently with celebrated matrix scaling algorithms, such as Sinkhorn's fixed point iteration method~\citep{sinkhorn1967,knight2008,kalantari2008}. 

Sinkhorn scaling for computing OT distances is a well studied problem in many recent works. 
The main idea is to improve the matrix-vector operations that are the computational bottleneck of Sinkhorn’s algorithm. 
\citet{altschulernips17} proposed a greedy version of Sinkhorn, called Greenkhorn algorithm, allowing to select columns and rows to be updated that most violate the constraints.                   
Another approach based on low-rank approximation of the cost matrix using Nystrom method induces the Nys-Sink algorithm~\citep{altschuler2018Nystrom}. 
Other classical optimization algorithms have been considered to approximate the OT, for instance accelerated gradient descent~\citep{xie2018proxpointOT,dvurechensky18aICML,lin2019}, quasi Newton methods~\citep{blondel2018ICML,cuturi2016SIAM} and stochastic gradient descent~\citep{genevay2016stochOT,khalilabid2018}. 

{In this paper, we propose a novel technique for accelerating the Sinkhorn algorithm when computing regularized OT distance between discrete measures. Our idea
is strongly related to a screening strategy when solving a \emph{Lasso}
problem in sparse supervised learning \citep{Ghaoui2010SafeFE}. Based on the fact
that a  transport plan resulting from an OT problem is sparse or presents a large
number of neglectable values \citep{blondel2018ICML}, our objective is to identify the  dual variables of an approximate Sinkhorn problem, that are smaller than a predefined threshold, and thus that can be safely removed before optimization.  
Within this global context, our contributions are the following :
\begin{itemize}
	  \setlength\itemsep{-0.1cm}
	
	\item From a methodological point of view, we propose a reformulation the dual of the Sinkhorn divergence problem by imposing variables to be larger than a threshold.
	This formulation allows us to introduce sufficient conditions, computable beforehand, for a variable to be strictly satisfied its constraint, leading then to
	a ``screened'' version of the dual of Sinkhorn divergence. 
	\item We provide some theoretical analysis of the solution of the ``screened''  Sinkhorn divergence, showing that its objective value and the marginal constraint satisfaction are properly controlled 	as the number of screened variables decreases.
	\item From an algorithmic standpoint, we use a constrained LBFGS algorithm \cite{nocedal1980,byrd1995L-BFGS-B} but provide a careful analysis of the lower bound and upper bound of the dual	variables, resulting in a well-posed and efficient algorithm denoted as \emph{Screenkhorn}.
	\item Our empirical analysis depicts how the approach behaves in a simple Sinkhorn divergence computation context. When considered in  complex machine learning
	pipelines, we show that \emph{Screenkhorn} can lead to strong gain in efficiency
	while not compromising on accuracy.
\end{itemize}}
%	
%
%These constraints are defined through a convex set which depends on two parameters acting like threshold and scaling factor.
%We prove that dual solution of this reformulation guarantees the existence of two active indices sets for the potential variables.
%Furthermore, the active sets are both directly linked to a priori fixed number budget of points from the supports of the given discrete measures.
%We then restrict the constraints feasibility by taking into account the properties providing by the active sets to get a ``screened'' version of the dual of Sinkhorn divergence. 
%Screenkhorn algorithm developed in this paper consists of two steps; the first one is an initialization step devoted to determine the active sets while the second is a constrained box-constrained L-BFGS-B solver, a limited-memory quasi-Newton algorithm~\cite{nocedal1980,byrd1995L-BFGS-B} that relies on an estimation of the inverse of the Hessian based on gradients differences. 
%L-BFGS-B algorithm was adapted to the OT setting via the dual of Sinkhorn divergence in ~\cite{cuturi2016SIAM,blondel2018ICML}.}

The remainder of the paper is organized as follow. In Section~\ref{sec:regularized_discrete_ot} we briefly review the basic setup of regularized discrete OT. 
Section~\ref{sec:screened_dual_of_sinkhorn_divergence} contains our main contribution, that is, the Screenkhorn algorithm. 
Section~\ref{sec:analysis_of_marginal_violations} devotes to theoretical guarantees for marginal violations of Screenkhorn. 
In Section~\ref{sec:numerical_experiments} we present numerical results for the proposed algorithm, compared with the state-of-art Sinkhorn algorithm as implemented in~\cite{flamary2017pot}. 
The proofs of theoretical results are postponed to the supplementary material.

\emph{Notation.} For any positive matrix $T \in \R^{n\times m}$, we define its negative entropy as $H(T) = -\sum_{i,j} T_{ij} \log(T_{ij}).$
Let $r(T) = T\mathbf 1_m \in \R^n$ and $c(T) = T^\top\mathbf 1_n \in \R^m$ denote the rows and columns sums of $T$ respectively. The coordinates $r_i(T)$ and $c_j(T)$ denote the $i$-th row sum and the $j$-th column sum of $T$, respectively.
The scalar product between two matrices denotes the usual inner product, that is $\inr{T, W} = \text{tr}(T^\top W) = \sum_{i,j}T_{ij}W_{ij},$ where $T^\top$ is the transpose of $T$. 
We write $\mathbf{1}$ (resp. $\mathbf{0}$) the vector having all coordinates equal to one (resp. zero).
$\Delta(w)$ denotes the diag operator, such that if $w \in \R^n$, then $\Delta(w) = \text{diag}(w_1, \ldots, w_n)\in \R^{n\times n}$.
For a set of indices $L=\{i_1, \ldots, i_k\} \subseteq \{1, \ldots, n\}$ satisfying $i_1 < \cdots <i_k,$ we denote the complementary set of $L$ by $L^\complement = \{1, \ldots, n\} \backslash L$. We also denote $|L|$ the cardinality of $L$.
Given a vector $w \in \R^n$, we denote $w_L= (w_{i_1}, \ldots, w_{i_k})^\top \in \R^k$ and its complementary $w_{L^\complement} \in \R^{n- k}$.  The notation is similar for matrices; given another subset of indices $S = \{j_1, \ldots, j_l\} \subseteq \{1, \ldots, m\}$ with $j_1 < \cdots <j_l,$ and a matrix $T\in \R^{n\times m}$, we use $T_{(L,S)}$, to denote the submatrix of $T$, namely the rows and columns of $T_{(L,S)}$ are indexed by $L$ and $S$ respectively.
When applied to matrices and vectors,  $\odot$ and $\oslash$ (Hadamard product and division) and exponential notations refer to elementwise operators.
Given two real numbers $a$ and $b$, we write $a\vee b = \max(a,b)$ and $a\wedge b = \min(a,b).$

% section introduction (end)
%!TEX root = main.tex

\section{Regularized discrete OT} % (fold)
\label{sec:regularized_discrete_ot}

We briefly expose in this section the setup of OT between two discrete measures. We then consider the case when those distributions are only available through a finite number of samples, that is $\mu = \sum_{i=1}^n \mu_i \delta_{x_i} \in \Sigma_n$ and $\nu = \sum_{j=1}^m \nu_i \delta_{y_j} \in \Sigma_m$, where $\Sigma_n$ is the probability simplex with $n$ bins, namely the set of probability vectors in $\R_+^n$, i.e., $\Sigma_n = \{w \in \R_+^n: \sum_{i=1}^n w_i = 1\}.$
We denote their probabilistic couplings set as $\Pi(\mu, \nu) = \{P \in \R_+^{n\times m}, P\mathbf{1}_m = \mu, P^\top \mathbf{1}_n = \nu\}.$ 
\paragraph{Sinkhorn divergence.}

Computing OT distance between the two discrete measures $\mu$ and $\nu$  amounts to solving a linear problem~\citep{kantorovich1942} given by
\begin{equation*}
  \label{monge-kantorovich}
  \mathcal{S}(\mu, \nu) =  \min_{P\in \Pi(\mu, \nu)} \inr{C, P},
\end{equation*}
where $P= (P_{ij}) \in \R^{n\times m}$ is called the transportation plan, namely each entry $P_{ij}$ represents the fraction of mass moving from $x_i$ to $y_j$, and $C= (C_{ij}) \in \R^{n\times m}$ is a cost matrix comprised of nonnegative elements and related to the energy needed to move a probability mass from $x_i$ to $y_j$. 
The entropic regularization of OT distances~\citep{cuturinips13} relies on the addition of a penalty term as follows:
\begin{equation}
\label{sinkhorn-primal}
  \mathcal{S}_\eta(\mu, \nu) =  \min_{P\in \Pi(\mu, \nu)} \{\inr{C, P} - \eta H(P)\},
\end{equation}
where $\eta > 0$ is a regularization parameter. We refer to $\mathcal{S}_\eta(\mu, \nu) $ as the \emph{Sinkhorn divergence}~\citep{cuturinips13}.

\paragraph{Dual of Sinkhorn divergence.}

Below we provide the derivation of the dual problem for the regularized OT problem~\eqref{sinkhorn-primal}. Towards this end, we begin with writing its Lagrangian dual function:
\begin{equation*}
  \mathscr{L}(P,w, z) = \inr{C,P} + \eta \inr{\log P, P} + \inr{w, P\mathbf{1}_m - \mu} + \inr{z,P^\top \mathbf{1}_n - \nu}.
\end{equation*}
The dual of Sinkhorn divergence can be derived by solving $\min_{P \in \R_+^{n\times m}}\mathscr{L}(P,w, z)$. It is easy to check that objective function $P\mapsto \mathscr{L}(P,w, z)$ is strongly convex and differentiable. Hence, one can solve the latter minimum by setting $\nabla_P \mathscr{L}(P,w, z)$ to $\mathbf{0}_{n\times m}$. Therefore, we get $  P^\star_{ij} = \exp\big(- \frac{1}{\eta} (w_i + z_j + C_{ij}) - 1\big), 
$ for all $i=1, \ldots, n$ and $j=1, \ldots, m$. Plugging this solution,  and setting the change of variables $u = -w/\eta - 1/2$ and $v = - z/\eta - 1/2$, the dual problem is given by
\begin{equation}
\label{sinkhorn-dual}
\min_{u \in \R^n, v\in\R^m}\big\{\Psi(u,v):= \mathbf{1}_n^\top B(u,v)\mathbf{1}_m - \inr{u, \mu} - \inr{v, \nu} \big\},
\end{equation}
where $B(u,v) := \Delta(e^{u}) K \Delta(e^{v})$ and $K := e^{-C/\eta}$ stands for the Gibbs kernel associated to the cost matrix $C$. 
We refer to problem~\eqref{sinkhorn-dual} as the \emph{dual of Sinkhorn divergence}. Then, the optimal solution $P^\star$ of the primal problem~\eqref{sinkhorn-primal} takes the form $P^\star = \Delta(e^{u^\star}) K \Delta(e^{v^\star})$
where the couple $(u^\star, v^\star)$ satisfies:
\begin{align*}
\label{sinkhorn-dual}
  (u^\star, v^\star) &= \argmin_{u \in \R^{n}, v\in \R^m} \{\Psi(u,v)\}.
\end{align*}
Note that the matrices $\Delta(e^{u^\star})$ and $\Delta(e^{v^\star})$ are unique up to a constant factor~\citep{sinkhorn1967}. Moreover, $P^\star$ can be solved efficiently by iterative Bregman projections~\citep{benamou2015IterativeBP} referred to as Sinkhorn iterations, and the method is referred to as \textsc{Sinkhorn} algorithm which, recently, has been proven to achieve a near-$\bigO(n^2)$ complexity~\citep{altschulernips17}.

% section regularized_discrete_ot (end)
\section{Screened dual of Sinkhorn divergence} % (fold)
\label{sec:screened_dual_of_sinkhorn_divergence}

% In this section we describe the main algorithm studied in this paper. 
For a fixed $\varepsilon > 0$ and $\kappa > 0$ we define an \emph{approximate dual of Sinkhorn divergence} as follows:
\begin{equation} 
\label{screen-sinkhorn}
\min_{u \in \mathcal{C}^n_{\frac \varepsilon \kappa}, v\in \mathcal{C}^m_{\varepsilon\kappa}} \big\{\Psi_{\kappa}(u,v):= \mathbf{1}_n^\top B(u,v)\mathbf{1}_m - \inr{\kappa u, \mu} - \inr{\frac v\kappa, \nu} \big\},
\end{equation}
where $\mathcal{C}^r_{\alpha} \subseteq \R^r$, for $r\in \mathbb N$ and $\alpha >0$, is a convex set  given by $\mathcal{C}^r_{\alpha} = \{w \in \R^{r}: \min_{1\leq i\leq r} e^{w_i} \geq \alpha\}$.

The objective function $\Psi_{\kappa}$ is convex with respect to $(u,v)$, then the set of optima of problem~\eqref{screen-sinkhorn} is non empty. 
The $\kappa$-parameter plays a role of scaling factor, namely it allows to get a closed order of the potential variables $e^u$ and $e^v$, while the $\varepsilon$-parameter acts like a threshold for $e^u$ and $e^v$.
Note that the approximate dual of Sinkhorn divergence coincides with the dual of Sinkhorn divergence~\eqref{sinkhorn-dual} in the setting of $\varepsilon=0$ and $\kappa=1$.
The screening procedure presented in this work is based on constructing two {active sets} $I_{\varepsilon, \kappa}$ and $J_{\varepsilon, \kappa}$ throughout the dual problem of~\eqref{screen-sinkhorn} in the following way:

\begin{lemma}
\label{lemma_actives_sets}
Let $(u^{*}, v^{*})$ be an optimal solution of the problem~\eqref{screen-sinkhorn}. 
Define
\begin{equation}
\label{I_epsilon_kappa_J_epsilon_kappa}
I_{\varepsilon,\kappa} = \big\{i=1, \ldots, n: \mu_i \geq \frac {\varepsilon^2} \kappa^{} r_i(K)\big\}, J_{\varepsilon,\kappa} = \big\{j=1, \ldots, m: \nu_j \geq \kappa{\varepsilon^2}{} c_j(K)\big\}
\end{equation}
Then one has $e^{u^{*}_i} = \varepsilon\kappa^{-1}$ and $e^{v^{*}_j} = \varepsilon\kappa$ for all $i \in I^\complement_{\varepsilon,\kappa} $ and $j\in J^\complement_{\varepsilon,\kappa} .$
\end{lemma}

First order conditions applied to $(u^{*}, v^{*})$ ensure that if $e^{u^{*}_i} > \varepsilon\kappa^{-1}$ then $e^{u^{*}_i} (Ke^{v^{*}})_i =  \kappa\mu_i$ and if $e^{v^{*}_j} > \varepsilon\kappa$ then $e^{v^{*}_j} (K^\top e^{u^{*}})_j =  \kappa^{-1}\nu_j$ 
which correspond to the Sinkhorn marginal conditions up to the scaling factor $\kappa$. 

\paragraph{Screening with a fixed number budget of points.}

Recall that the approximate dual of Sinkhorn divergence is defined with respect to $\varepsilon$ and $\kappa$. 
The explicit determination of its values depends on a priori \emph{fixed number budget of points} from the supports of $\mu$ and $\nu$. % in problem~\eqref{screen-sinkhorn}.  
In the sequel of the paper, we denote by $n_b \in\{1, \ldots, n\}$ and the $m_b\in\{1, \ldots, m\}$ the number budget of points to be given for resolving problem~\eqref{screen-sinkhorn}. 

Let us define $\xi \in \R^n$ and $\zeta \in \R^m$ to be the ordered decreasing vectors of $\mu \oslash r(K)$ and $\nu \oslash c(K)$ respectively, that is $\xi_1 \geq \xi_2 \geq \cdots \geq \xi_n$ and $\zeta_1 \geq \zeta_2 \geq \cdots \geq \zeta_m$.
To keep only $n_b$-budget and $m_b$-budget of points, the parameters $\kappa$ and $\varepsilon$ satisfy ${\varepsilon^2}\kappa^{-1} = \xi_{n_b}$ and $\varepsilon^2\kappa = \zeta_{m_b}$. Hence 
\begin{equation}
\label{epsilon_kappa}
 \varepsilon = (\xi_{n_b}\zeta_{m_b})^{1/4} \text{ and } \kappa = \sqrt{\frac{\zeta_{m_b}}{\xi_{n_b}}}.
\end{equation}
Note that $|I_{\varepsilon, \kappa}| = n_b$ and $|J_{\varepsilon, \kappa}| = m_b$. 
Using the previous analysis, any solution $(u^*, v^*)$ of problem~\eqref{screen-sinkhorn} satisfy $e^{u^*_i} \geq \varepsilon\kappa^{-1}$ and $e^{v^*_j} \geq \varepsilon\kappa$ for all $(i,j) \in (I_{\varepsilon,\kappa}\times J_{\varepsilon,\kappa}),$ and $e^{u^*_i} = \varepsilon\kappa^{-1}$ and $e^{v^*_j} = \varepsilon\kappa$ for all $(i,j) \in (I^\complement_{\varepsilon,\kappa}\times J^\complement_{\varepsilon,\kappa})$.

Basing on that facts we restrict the constraints feasibility $\mathcal{C}^n_{\frac \varepsilon \kappa} \cap \mathcal{C}^m_{\varepsilon\kappa}$ in problem~\eqref{screen-sinkhorn} to the screened domain $\mathcal{U}_{\text{sc}} \cap \mathcal{V}_{\text{sc}}$ where 
\begin{equation*}
\mathcal{U}_{\text{sc}} = \{u \in \R^n: e^{u_{I_{\varepsilon,\kappa}}} \succeq \frac \varepsilon\kappa\mathbf 1_{n_b}, \text{ and } e^{u_{I^\complement_{\varepsilon,\kappa}}} = \frac\varepsilon\kappa\mathbf 1_{n - n_b}\},
\end{equation*}
and 
\begin{equation*}
	\mathcal{V}_{\text{sc}} =\{v\in\R^m: e^{v_{J_{\varepsilon,\kappa}}} \succeq \varepsilon\kappa \mathbf{1}_{m_b}, \text{ and } e^{v_{J^\complement_{\varepsilon,\kappa}}} = \varepsilon\kappa \mathbf 1_{m- m_b}\}.
\end{equation*}
where the vector comparison $\succeq$ has to be understood elementwise.
Now, we are ready to define the \emph{screened dual of Sinkhorn divergence} as 
\begin{align}
\label{screen-sinkhorn_second_def}
\min_{u \in \mathcal{U}_{\text{sc}}, v \in \mathcal{V}_{\text{sc}}}\{\Psi_{\varepsilon, \kappa}(u,v)\}
\end{align}
where 
\begin{align*} 
\Psi_{\varepsilon,\kappa}(u, v) &= (e^{u_{I_{\varepsilon,\kappa}}})^\top K_{(I_{\varepsilon,\kappa}, J_{\varepsilon,\kappa})} e^{v_{J_{\varepsilon,\kappa}}} + 
\varepsilon \kappa (e^{u_{I_{\varepsilon,\kappa}}})^\top K_{(I_{\varepsilon,\kappa}, J^\complement_{\varepsilon,\kappa})}\mathbf 1_{m_b} + \varepsilon \kappa^{-1} \mathbf 1_{n_b}^\top K_{(I^\complement_{\varepsilon,\kappa}, J_{\varepsilon,\kappa})}e^{v_{J_{\varepsilon,\kappa}}}\\
&\qquad - \kappa \mu_{I_{\varepsilon,\kappa}}^\top u_{I_{\varepsilon,\kappa}} - \kappa^{-1} \nu_{J_{\varepsilon,\kappa}}^\top v_{J_{\varepsilon,\kappa}} + \Xi
\end{align*}
with $\Xi = \varepsilon^2 \sum_{i \in I^\complement_{\varepsilon,\kappa}, j \in J^\complement_{\varepsilon,\kappa}} K_{ij} -\kappa \log(\varepsilon\kappa^{-1})\sum_{i \in I^\complement_{\varepsilon,\kappa}}\mu_i - \kappa^{-1} \log(\varepsilon\kappa)\sum_{j\in J^\complement_{\varepsilon,\kappa}} \nu_j$.

Pseudocode of our proposed algorithm is given in Algorithm~\ref{screenkhorn}.
Screenkhorn consists of two steps: the first one is an initialization where we calculate the active sets $I_{\varepsilon,\kappa}$, $J_{\varepsilon,\kappa}$. 
The second is a constrained L-BFGS solver~\cite{zhu1997-LBFGS} for the stacked variable $\theta=(u_{I_{\varepsilon,\kappa}},v_{J_{\varepsilon,\kappa}}).$ 
% It is worth to note that the couple variables $(u,v)$ to be optimized in Screenkhorn belongs to $\R^{n_b\times m_b}$. 
It is worth to note that Screenkhorn uses only the restricted parts $K_{(I_{\varepsilon,\kappa}, J_{\varepsilon,\kappa})},$ $K_{(I_{\varepsilon,\kappa}, J^\complement_{\varepsilon,\kappa})},$ and $K_{(I^\complement_{\varepsilon,\kappa}, J_{\varepsilon,\kappa})}$ of the Gibbs kernel $K$, in contrast to Sinkhorn algorithm which performs alternating updates of all rows and columns of $K.$

The following lemma expresses upper and lower bounds to be respected in Screenkhorn.
\begin{lemma}
\label{lemma_bounds_of_usc_and_vsc}
Let $(u^{\text{sc}}, v^{\text{sc}})$ be an optimal solution of problem~\eqref{screen-sinkhorn_second_def}. Then,
one has
\begin{equation}
\label{bound_on_u}
\frac \varepsilon\kappa \vee \frac{\min_{i \in I_{\varepsilon,\kappa}}\mu_i}{\varepsilon (m- m_b) + \varepsilon \vee \frac{\max_{j\in J_{\varepsilon,\kappa}} \nu_j}{n\varepsilon\kappa\min_{i,j}K_{ij}} m_b} \leq e^{u^{\text{sc}}_i} \leq \frac \varepsilon\kappa\vee \frac{\max_{i \in I_{\varepsilon,\kappa}} \mu_i}{m\varepsilon\min_{i,j}K_{ij}},
\end{equation}
and
\begin{equation}
\label{bound_on_v}
\varepsilon\kappa \vee \frac{\min_{j \in J_{\varepsilon,\kappa}}\nu_j}{\varepsilon(n- n_b) + \varepsilon \vee \frac{\kappa\max_{i\in I_{\varepsilon,\kappa}} \mu_i}{m\varepsilon\min_{i,j}K_{ij}} n_b} \leq e^{v^{\text{sc}}_j} \leq \varepsilon\kappa \vee \frac{\max_{j \in J_{\varepsilon,\kappa}} \nu_j}{n\varepsilon\min_{i,j}K_{ij}}
\end{equation}
for all $i\in I_{\varepsilon,\kappa}$ and $j\in J_{\varepsilon,\kappa}$.
\end{lemma}

\LinesNotNumbered
\begin{algorithm}[htbp]
\SetNlSty{textbf}{}{.}
\DontPrintSemicolon
\caption{Screenkhorn$(C,\eta,\mu,\nu,n_b,m_b)$}
\label{screenkhorn}
% \textbf{input: }{$C$, $\eta$, $\mu \in \Sigma_n$, $\nu \in \Sigma_m$, $n_b$ and $m_b$;}\\

\textbf{step 1:} \textcolor{black}{Initialization}\\

% \nl   $K \gets \exp(-C/\eta);$\\
\nl   $\xi \gets \mu \oslash r(K);$ \\
\nl   $\zeta \gets \nu \oslash c(K);$\\
\nl   $\xi \gets \texttt{sort}(\xi);$ //(decreasing order)\\
\nl   $\zeta \gets \texttt{sort}(\zeta);$ //(decreasing order)\\
\nl   $\varepsilon \gets (\xi_{n_b}\zeta_{m_b})^{1/4}, \text{  } \kappa \gets \sqrt{{\zeta_{m_b}}/{\xi_{n_b}}}$;\\
\nl   $I_{\varepsilon,\kappa} \gets \{i=1, \ldots, n: \mu_i \geq {\varepsilon^2} \kappa^{-1} r_i(K)\};$\\
\nl   $J_{\varepsilon,\kappa} \gets \{j=1, \ldots, m: \nu_j \geq \varepsilon^2\kappa c_j(K)\};$\\ 

% \textbf{step 2:} \textcolor{black}{Bounds for L-BFGS}

\nl   $K_{\min} \gets \min_{i \in I_{\varepsilon,\kappa}, j \in J_{\varepsilon,\kappa}}K_{ij};$ \\
\nl   $\underline{\mu} \gets \min_{i \in I_{\varepsilon,\kappa}} \mu_i, \bar{\mu} \gets \max_{i \in I_{\varepsilon,\kappa}} \mu_i$; \\
\nl   $\underline{\nu} \gets \min_{j \in J_{\varepsilon,\kappa}} \mu_i, \bar{\nu} \gets \max_{j \in J_{\varepsilon,\kappa}} \mu_i$; \\
\nl   $\underline{u} \gets \log\big(\frac \varepsilon\kappa \vee \frac{\underline{\mu}}{\varepsilon (m-m_b) + \varepsilon \vee \frac{\bar{\nu}}{n\varepsilon\kappa K_{\min}} m_b}\big), \bar{u} \gets  \log\big(\frac \varepsilon\kappa\vee \frac{\bar{\mu}}{m\varepsilon K_{\min}}\big);$\\
\nl   $\underline{v} \gets \log\big(\varepsilon\kappa \vee \frac{\underline{\nu}}{\varepsilon(n-n_b) + \varepsilon \vee \frac{\kappa\bar{\mu}}{m\varepsilon K_{\min}} n_b}\big), \bar{v} \gets \log\big(\varepsilon\kappa \vee \frac{\bar{\nu}}{n\varepsilon K_{\min}}\big);$\\
\nl   $ \bar{\theta} \gets \text{stack}(\bar{u}\mathbf 1_{n_b}, \bar{v}\mathbf 1_{m_b});$\\
\nl   $ \underline{\theta} \gets \text{stack}(\underline{u}\mathbf 1_{n_b}, \underline{v}\mathbf 1_{m_b}) ;$\\ %\in \R^{n_b\times m_b}

\textbf{step 2:} \textcolor{black}{L-BFGS}\\

\nl  $u^{(0)} \gets \log(\varepsilon\kappa^{-1}) \mathbf 1_{n_b} ;$\\
\nl  $v^{(0)} \gets \log(\varepsilon\kappa) \mathbf 1_{m_b} ;$\\
\nl  $\theta^{(0)} \gets \text{stack}[u^{(0)}, v^{(0)}];$\\
\nl   $\theta \gets \text{L-BFGS}(\theta^{(0)}, \underline{\theta}, \bar{\theta});$\\
\nl   $\theta_u \gets (\theta_1, \ldots, \theta_{n_b})^\top, \theta_v \gets(\theta_{n_b+1}, \ldots, \theta_{n_b+m_b})^\top;$\\

% \textbf{step 4:} \textcolor{black}{Output}\\

\nl   {$u^{sc}_i \gets (\theta_u)_i$ if $i \in I_{\varepsilon,\kappa}$ and $u_i \gets \log(\varepsilon\kappa^{-1})$ if $i \in I^\complement_{\varepsilon,\kappa};$}\\
\nl   {$v^{sc}_j \gets (\theta_v)_j$ if $j \in J_{\varepsilon,\kappa}$ and $v_j \gets \log(\varepsilon\kappa)$ if $j \in J^\complement_{\varepsilon,\kappa};$}\\
\nl   \Return{$B(u^{\text{sc}},v^{\text{sc}})$.}
\end{algorithm}

% section screened_dual_of_sinkhorn_divergence (end)
%!TEX root = main.tex

\section{Analysis of marginal violations} % (fold)
\label{sec:analysis_of_marginal_violations}

This section is devoted to establishing theoretical guarantees for the marginal violations of Screenkhorn. 
We first define the screened marginals $\mu^{\text{sc}} = B(u^{\text{sc}}, v^{\text{sc}}) \mathbf 1_m$ and $\nu^{\text{sc}} = B(u^{\text{sc}}, v^{\text{sc}})^\top \mathbf 1_n.$ 
Lemma~\ref{lemma_bounds_on_marginals} expresses an upper bound with respect to $\ell_1$-norm of $\mu^{\text{sc}}$and $\nu^{\text{sc}}$. % while Proposition~\ref{proposition_error_in_marginals} gives also an upper bound of the marginal errors.

\begin{lemma}
\label{lemma_bounds_on_marginals}
Let $(u^{\text{sc}}, v^{\text{sc}})$ be an optimal solution of problem~\eqref{screen-sinkhorn_second_def}.
Then one has 
\begin{equation*}
\norm{\mu^{\text{sc}}}_1 \leq \kappa \norm{\mu_{I_{\varepsilon,\kappa}}^{\text{}}}_1 + (n-n_b) \Big(\frac{m_b\max_{j \in J_{\varepsilon,\kappa}} \nu_j}{n\kappa K_{\min}} + (m-m_b)\varepsilon^2 \Big)
\end{equation*}
and 
\begin{equation*}
\norm{\nu^{\text{sc}}}_1 \leq \kappa^{-1} \norm{\nu_{J_{\varepsilon,\kappa}}}_1 + (m-m_b)\Big(\frac{n_b\kappa \max_{i\in I_{\varepsilon,\kappa}}\mu_i}{m K_{\min}} + (n-n_b)\varepsilon^2\Big).
\end{equation*}
\end{lemma}
The following Proposition gives also an upper bound of the marginal errors.
\begin{proposition}
\label{proposition_error_in_marginals}
One has 
\begin{align*} %{\norm{{\mu}^{\text{sc}}}_1} 
{\norm{{\mu} -{\mu}^{\text{sc}}}^2_1}&\leq 7\norm{\mu^{\text{sc}}}_1 n_b(\kappa-\log(\kappa)-1)\max_{i} \mu_i + 7\norm{\mu^{\text{sc}}}_1  (n-n_b)\bigg(\frac{m_b\max_{j}\nu_j}{n\kappa K_{\min}} + (m- m_b) \varepsilon^2\\
&\qquad - \min_{i}\mu_i + \max_{i} \mu_i\log\Big(\frac{\kappa(n-n_b+ 1)\max_{i} \mu_i}{m_b K_{\min}\min_{j \in J_{\varepsilon,\kappa}}\nu_j} + \frac{\kappa^2\max_{i} \mu_i}{mm_b\varepsilon^2(K_{\min})^2\min_{j \in J_{\varepsilon,\kappa}}\nu_j}\Big)\bigg)% \bigg\}^{1/2}
\end{align*}
and 
\begin{align*} %{\norm{{\nu}^{\text{sc}}}_1}
{\norm{{\nu} -{\nu}^{\text{sc}}}^2_1} &\leq 7\norm{\nu^{\text{sc}}}_1m_b(\kappa-\log(\kappa)-1)\max_{j} \nu_j + 7\norm{\nu^{\text{sc}}}_1(m-m_b)\bigg(\frac{n_b\kappa\max_{i}\mu_i}{n K_{\min}} + (n- n_b) \varepsilon^2 - \min_{j}\nu_j\\
&+ \max_{j} \nu_j\log\Big(\frac{\kappa(m-m_b+ 1)\max_{j} \nu_j}{n_b K_{\min}\min_{i \in I_{\varepsilon,\kappa}}\mu_i} + \frac{\kappa^2\max_{j} \nu_j}{nn_b\varepsilon^2(K_{\min})^2\min_{i \in I_{\varepsilon,\kappa}}\mu_i}\Big).% \bigg\}^{1/2}
\end{align*}

\end{proposition}

% section analysis_of_marginal_violations (end)

%!TEX root = main.tex

\section{Numerical experiments} % (fold)
\label{sec:numerical_experiments}

In this section, we present some numerical analyses of our
\emph{Screenkhorn} algorithm and show how it behaves when
integrated into some complex machine learning pipelines.

\subsection{Setup}

We have implemented our \emph{Screenkhorn} algorithm in Python and used the L-BFGS-B of
Scipy. Regarding the machine-learning based comparison, we have based our code
on the ones of Python Optimal Transport toolbox (POT)~\citep{flamary2017pot} and just replaced the Sinkhorn function call with a \emph{Screenkhorn} one. We have considered the POT's default Sinkhorn stopping criterion parameters and for \emph{Screenkhorn}, the L-BFGS-B algorithm is stopped when the 
largest component of the projected gradient is smaller than $10^{-6}$, when the number of iterations of number of objective function evaluations reach $10^{5}$. For all applications, we have set $\eta=1$ unless otherwise specified.

\subsection{Analysing on toy problem}

We compare \emph{Screenkhorn} to Sinkhorn as implemented in Python Optimal Transport toolbox\footnote{\url{https://pot.readthedocs.io/en/stable/index.html}} on  a synthetic example. The dataset we use consists of source samples generated from a bi-dimensional gaussian mixture and target samples following the same distribution but with different gaussian means. We consider an unsupervised domain adaptation using optimal transport with entropic regularization.  Several settings are explored: different values of $\eta$, the regularization parameter, the allowed budget $\frac{n_b}{n} = \frac{m_b}{m} = $ ranging from $0.01$ to $0.99$, different values of $n$ and $m$.
% and, whether the cost matrix $C$ is normalized or not.
 We empirically measure  marginal violations as the norms $\norm{{\mu} -{\mu}^{\text{sc}}}_1$ and $\norm{{\nu} -{\nu}^{\text{sc}}}_1$, running time expressed as $\frac{T_{\text{\emph{Screenkhorn}}}}{T_{\text{Sinkhorn}}}$ and the relative divergence difference $| \inr{C, P^\star} - \inr{C, P^{\text{sc}}}|/\inr{C, P^\star}$ between \emph{Screenkhorn} and Sinkhorn, where $P^\star = \Delta(e^{u^\star}) K \Delta(e^{v^\star})$ and $P^{\text{sc}} = \Delta(e^{u^{\text{sc}}}) K \Delta(e^{v^{\text{sc}}}).$
Figure \ref{fig:margin_expe} summarizes the observed behaviors of both algorithms under these settings. We choose to only report results for $n=m=1000$ as we get similar findings for other values of $n$ and $m$. 

\begin{figure*}[t]
	\begin{center}
		\includegraphics[width=0.35\textwidth]{./figs/norm_M_Mu_marginals_toy_n1000}\hspace{2cm}
		\includegraphics[width=0.35\textwidth]{./figs/norm_M_Nu_marginals_toy_n1000} \\
		\includegraphics[width=0.35\textwidth]{./figs/norm_M_time_toy_n1000}\hspace{2cm}
		\includegraphics[width=0.35\textwidth]{./figs/norm_M_div_toy_n1000}
	\end{center}
	\caption{Empirical evaluation of \emph{Screenkhorn} vs Sinkhorn for normalized cost matrix. Top panel: marginal violations in relation with the budget of points. Bottom panel: (left) ratio of computation times    $\frac{T_{\text{Sinkhorn}}}{T_{\text{\emph{Screenkhorn}}}}$ and, (right) relative divergence variation. The results are averaged over $30$ trials.} 
		\label{fig:margin_expe}
\end{figure*}
%
\emph{Screenkhorn} provides good approximation of the marginals $\mu$ and $\nu$ for ``high'' values of the regularization parameter $\eta$ ($\eta > 1$). The approximation quality diminishes for small $\eta$. As expected $\norm{{\mu} -{\mu}^{\text{sc}}}_1$ and $\norm{{\nu} -{\nu}^{\text{sc}}}_1$ converge towards zero when increasing the budget of points. Remarkably marginal violations are almost negligible whatever the budget for high $\eta$.  According to computation gain, \emph{Screenkhorn} is almost  2 times faster than Sinkhorn at high decimation factor $n/n_b$ (low budget) while the reverse holds when $n/n_b$ gets close to 1.  Computational benefit of \emph{Screenkhorn} also depends on $\eta$ with appropriate values $\eta \leq 1$. Finally except for $\eta=0.1$ \emph{Screenkhorn} achieves a  divergence $\inr{C, P}$ close to the one of Sinkhorn showing that our static screening test does not harm Sinkhorn divergence.  \emph{Screenkhorn} will be practically useful in cases when modest accuracy on the divergence is sufficient as a loss function a gradient descent method optimizes over (see next section).


\subsection{Integrating \emph{Screenkhorn} into machine learning pipelines}

Here, we analyse the impact of using \emph{Screenkhorn}
instead of Sinkhorn in a complex machine learning pipeline. Our two applications
are a dimensionality reduction technique, denoted as Wasserstein Discriminant Analysis (WDA), based on Wasserstein distance approximated
through Sinkhorn divergence \citep{flamary2018WDA} and a domain-adaptation using optimal transport mapping \citep{courty2017optimal}, named OTDA. 

WDA aims at finding a linear projection which minimize the ratio of distance between intra-class samples and distance inter-class samples, where the distance is understood
in a Sinkhorn divergence sense. We have used a toy problem involving Gaussian classes with $2$ discriminative features and $8$ noisy features and the MNIST dataset. For the
former problem, we aim at find the best two-dimensional linear subspace in a WDA sense whereas for MNIST, we look for a subspace of dimension $20$ starting from the original
$728$ dimensions.  Quality of the retrieved subspace are evaluated using classification task based on a $1$-nearest neighbour approach.

Figure \ref{fig:wda} presents the average gain (over $30$ trials) in computational time we get as the number of examples evolve and for different decimation factors of the \emph{Screenkorn} problem.
Analysis of the quality of the subspace have been deported to the supplementary material (see Figure 1), but we can remark a small loss of performance of \emph{Screenkhorn} for the toy problem, while
for MNIST, accuracies are equivalent regardless of the decimation factor.  We can note
that for the minimal gains are respectively $2$ and $4.5$ for the toy and MNIST problem
whereas the maximal gain for $4000$ samples is slightly larger than an order of magnitude. 

\begin{figure*}[t]
	\centering
%	\includegraphics[width=4.cm]{./figs/wda_accur_toy.pdf}
%	\includegraphics[width=4.cm]{../figs/wda_accur_mnist.pdf}
	\includegraphics[width=0.39\textwidth]{./figs/wda_gain_toy.pdf}\hspace{1.5cm}
	\includegraphics[width=0.39\textwidth]{./figs/wda_gain_mnist.pdf}
	\caption{Wasserstein Discriminant Analysis : running time gain for a toy dataset and for MNIST as a function of the number of examples and the data decimation factor in \emph{Screenkhorn}.}
	\label{fig:wda}
\end{figure*}
\begin{figure*}[t]
	\centering

	\includegraphics[width=0.43\textwidth]{./figs/da_gain_mnist_regcl1.pdf}\hspace{1cm}
	\includegraphics[width=0.43\textwidth]{./figs/da_gain_mnist_regcl10.pdf}
	\caption{OT Domain adaptation : running time gain for a toy dataset and for MNIST as a function of the number of examples and the data decimation factor in \emph{Screenkhorn}.}
	\label{fig:otda}
\end{figure*}

For the optimal transport based domain adaptation problem, we have considered the
OTDA with $\ell_{0.5,1}$ group-lasso regularizer that helps in exploiting available labels in the source domain. The problem is solved using an majorization-minimization approach 
for handling the non-convexity of the problem. Hence, at each iteration, a Sinkhorn/\emph{Screenkhorn} has to be computed. As a domain-adaptation problem, we have
used a MNIST to USPS problem in which features have been extracted from the
feature extractor of an domain adversarial neural networks \citep{ganin2016domain} before full convergence of the networks (so as to leave room for OT adaptation). 
Figure \ref{fig:otda} reports the gain in running time for $2$ different values
of the group-lasso regularizer hyperparameter, while the curves of performances are
reported in the supplementary material. We can note that for all the  \emph{Screenkhorn} with different decimation factors, the gain in computation goes from a factor of $4$ to $12$, while accuracies are somewhat equivalent.

% section numerical_experiments (end)

\section{Conclusion}
The paper introduces a novel efficient approximation of the Sinkhorn divergence
based on a screening strategy. Screening some of the Sinkhorn dual variables
has been made possible by defining a novel constrained dual problem and by 
carefully analyzing its optimality conditions. From the latter, we derived some
sufficient conditions depending on the ground cost matrix, that some dual variables are smaller than a given threshold. Hence, we need just to solve a restricted
dual Sinkhorn problem using an off-the-shelf LBFGS algorithm. We also provided
some theoretical guarantees of the quality of the approximation with respect to
the number of variables that has been screened. Numerical experiments show 
the behaviour of our \emph{Screenkhorn} algorithm and computational time gain it can
achieve when integrated in some complex machine learning pipelines.

% \subsubsection*{Acknowledgments}

% Use unnumbered third level headings for the acknowledgments. All acknowledgments go at the end of the paper. Do not include acknowledgments in the anonymized submission, only in the final paper.

\small
\bibliography{biblio}
\bibliographystyle{chicago}
%well as with those for harvard, apalike, chicago, astron, authordate, and of course natbib.

\newpage
\section{Supplementary material}
%!TEX root = main.tex
\subsection{Proof of Lemma~\ref{lemma_actives_sets}}

Since the objective function $\Psi_{\kappa}$ is convex with respect to $(u,v)$, the set of optima of problem~\eqref{screen-sinkhorn} is non empty.
Introducing two dual variables $\lambda \in \R^n_{+}$ and $\beta \in \R^m_{+}$ for each constraint, the Lagrangian of problem~\eqref{screen-sinkhorn} reads as 
\begin{equation*}
  \mathscr{L}(u,v, \lambda, \beta) = \frac \varepsilon\kappa\inr{\lambda, \mathbf{1}_n} + \varepsilon\kappa\inr{\beta, \mathbf{1}_m} + \mathbf{1}_n^\top B(u,v) \mathbf{1}_m - \inr{\kappa u, \mu} - \inr{\frac v\kappa, \nu} -\inr{\lambda,e^{u}} - \inr{\beta,e^{v}}
\end{equation*}
First order conditions then yield that the Lagrangian multiplicators solutions $\lambda^{*}$ and $\beta^{*}$ satisfy 
\begin{align*}
  &\nabla_u\mathscr{L}(u^{*},v^{*}, \lambda^{*}, \beta^{*})=  e^{u^{*}} \odot(Ke^{v^{*}} - \lambda^{*}) - \kappa\mu = \mathbf 0_n,\\
  & \text{ and } \nabla_v\mathscr{L}(u^{*},v^{*}, \lambda^{*}, \beta^{*})=  e^{v^{*}} \odot(K^\top e^{u^{*}} - \beta) - \frac \nu\kappa = \mathbf 0_m
\end{align*}
which leads to 
\begin{align*}
  &\lambda^{*} = K e^{v^{*}} - \kappa\mu \oslash e^{u^{*}} \text{ and }
  \beta^{*} = K^\top e^{u^{*}} - \nu \oslash \kappa e^{v^{*}}
\end{align*}

For all $i=1, \ldots, n$ we have that $e^{u^{*}_i} \geq \varepsilon\kappa^{-1}$. Further, the condition on the dual variable $\lambda^{*}_i > 0$  ensures that $e^{u^{*}_i} = \varepsilon\kappa^{-1}$ and hence $i \in I^\complement_{\varepsilon,\kappa}$. We have that $\lambda^{*}_i > 0$ is equivalent to $e^{u^{*}_i}r_i(K) e^{v^{*}_j} >  \kappa{\mu_i}$ which  is satisfied when $\varepsilon^2r_i(K) >  \kappa{\mu_i}.$  
In a symmetric way we can prove the same statement for $e^{v^{*}_j}$.

\subsection{Proof of Proposition~\ref{prop:bounds_of_usc_and_vsc}}

We prove only the first statement~\eqref{bound_on_u} and similarly we can prove the second one~\eqref{bound_on_v}.
For all $i\in I_{\varepsilon,\kappa}$, we have $e^{u^{\text{sc}}_i} > \frac \varepsilon\kappa$ or $e^{u^{\text{sc}}_i} = \frac \varepsilon\kappa$. In one hand, if $e^{u^{\text{sc}}_i} > \frac \varepsilon\kappa$ then according to the optimality conditions $\lambda^{\text{sc}}_i = 0,$ which implies $e^{u^{\text{sc}}_i} \sum_{j=1}^m K_{ij} e^{v^{\text{sc}}_j} = \kappa\mu_i$.
In another hand, we have 
\begin{align*}
e^{u^{\text{sc}}_i} \min_{i,j}K_{ij} \sum_{j=1}^m e^{v^{\text{sc}}_j} \leq e^{u^{\text{sc}}_i} \sum_{j=1}^m K_{ij} e^{v^{\text{sc}}_j} = \kappa\mu_i.
\end{align*}
We further observe that $\sum_{j=1}^m e^{v^{\text{sc}}_j} = \sum_{j \in J_{\varepsilon,\kappa}} e^{v^{\text{sc}}_j} + \sum_{j \in J^\complement_{\varepsilon,\kappa}} e^{v^{\text{sc}}_j} \geq \varepsilon\kappa |J_{\varepsilon,\kappa}| + \varepsilon\kappa |J^\complement_{\varepsilon,\kappa}|=\varepsilon\kappa m.$ Then
\begin{equation*}
\max_{i\in I_{\varepsilon,\kappa}} e^{u^{\text{sc}}_i} \leq \frac \varepsilon\kappa \vee \frac{\max_{i\in I_{\varepsilon,\kappa}}\mu_i}{m\varepsilon K_{\min}} \leq \frac \varepsilon\kappa \vee \frac{\max_{i\in I_{\varepsilon,\kappa}}\mu_i}{m\varepsilon K_{\min}}.
\end{equation*}
Analogously, one can obtain for all $j\in J_{\varepsilon,\kappa}$
\begin{equation}
\label{upper_bound_v_potential}
\max_{j\in J_{\varepsilon,\kappa}}e^{v^{\text{sc}}_j} \leq \varepsilon\kappa \vee \frac{\max_{j \in J_{\varepsilon,\kappa}} \nu_j}{n\varepsilon K_{\min}} \leq \varepsilon\kappa \vee \frac{\max_{j \in J_{\varepsilon,\kappa}} \nu_j}{n\varepsilon K_{\min}} .
\end{equation}

Now, since $K_{ij} \leq 1$, we have 
\begin{align*}
e^{u^{\text{sc}}_i} \sum_{j=1}^m e^{v^{\text{sc}}_j} \geq e^{u^{\text{sc}}_i} \sum_{j=1}^m K_{ij}e^{v^{\text{sc}}_j} = \kappa\mu_i.
\end{align*}
Using~\eqref{upper_bound_v_potential}, we get 
\begin{align*}
\sum_{j=1}^m e^{v^{\text{sc}}_j} &= \sum_{j \in J_{\varepsilon,\kappa}} e^{v^{\text{sc}}_j} + \sum_{j \in J^\complement_{\varepsilon,\kappa}} e^{v^{\text{sc}}_j}
\leq \varepsilon\kappa |J^\complement_{\varepsilon,\kappa}| + \varepsilon\kappa \vee \frac{\max_{j\in J_{\varepsilon,\kappa}} \nu_j}{n\varepsilon K_{\min}} |J_{\varepsilon,\kappa}|.
\end{align*}
Therefore,
\begin{align*}
\min_{i \in I_{\varepsilon,\kappa}} e^{u^{\text{sc}}_i}  \geq \frac \varepsilon\kappa \vee \frac{\kappa\min_{I_{\varepsilon,\kappa}}\mu_i}{\varepsilon\kappa (m-m_b) + \varepsilon\kappa \vee \frac{\max_{j\in J_{\varepsilon,\kappa}} \nu_j}{n\varepsilon K_{\min}} m_b}.
\end{align*}

\subsection{Proof of Lemma~\ref{lemma_bounds_on_marginals}}

The optimality condition for $({u}^{\text{sc}}, {v}^{\text{sc}})$ entails 
\begin{align}
\label{i-th-marginal-mu} 
{\mu}^{\text{sc}}_i  &= 
\begin{cases}
e^{u^{\text{sc}}_i} \sum_{j=1}^m K_{ij} e^{v^{\text{sc}}_j}, \text{ if  }i \in I_{\varepsilon,\kappa},\\
\frac \varepsilon\kappa\sum_{j=1}^m K_{ij} e^{v^{\text{sc}}_j}, \text{ if  }i \in I^\complement_{\varepsilon,\kappa}
\end{cases}
=\begin{cases}
\kappa \mu_i, \text{ if  }i \in I_{\varepsilon,\kappa},\\
\frac \varepsilon\kappa\sum_{j=1}^m K_{ij} e^{v^{\text{sc}}_j}, \text{ if  }i \in I^\complement_{\varepsilon,\kappa},
\end{cases}
\end{align}
and 
\begin{align}
\label{i-th-marginal-nu} 
{\nu}^{\text{sc}}_j  &= 
\begin{cases}
e^{v^{\text{sc}}_j} \sum_{i=1}^n K_{ij} e^{u^{\text{sc}}_i}, \text{ if  }j \in J_{\varepsilon,\kappa},\\
\varepsilon\kappa\sum_{i=1}^n K_{ij} e^{u^{\text{sc}}_i}, \text{ if  }j \in J^\complement_{\varepsilon,\kappa}
\end{cases}
=\begin{cases}
\frac{\nu_j}{\kappa}, \text{ if  }j \in J_{\varepsilon,\kappa},\\
\varepsilon\kappa\sum_{i=1}^n K_{ij} e^{u^{\text{sc}}_i}, \text{ if  }j \in J^\complement_{\varepsilon,\kappa}.
\end{cases}
\end{align}

Using inequality~\eqref{bound_on_v}, we obtain 
\begin{align*}
\norm{\mu^{\text{sc}}}_1 &= \sum_{i \in I_{\varepsilon,\kappa}} \mu^{\text{sc}}_i +  \sum_{i \in I^\complement_{\varepsilon,\kappa}}\mu^{\text{sc}}_i\\
& \overset{\eqref{i-th-marginal-mu}}{=} \kappa \norm{\mu_{I_{\varepsilon,\kappa}}^{\text{sc}}}_1 + \frac \varepsilon\kappa \sum_{i \in I^\complement}\Big( \sum_{j \in J_{\varepsilon,\kappa}} K_{ij} e^{v^{\text{sc}}_j} + \varepsilon\kappa \sum_{j\in J^\complement_{\varepsilon,\kappa}}K_{ij}\Big)\\
& \overset{\eqref{bound_on_v}}{\leq} \kappa \norm{\mu_{I_{\varepsilon,\kappa}}^{\text{sc}}}_1 + (n-n_b) \Big(\frac{m_b\max_{j \in J_{\varepsilon,\kappa}} \nu_j}{n\kappa K_{\min}} + (m-m_b)\varepsilon^2 \Big).
\end{align*}
Again by left-hand-side of inequaltiy~\eqref{bound_on_v}, we arrive at 
\begin{align*}
\norm{\mu^{\text{sc}}}_1 %&= \sum_{i \in I_{\varepsilon,\kappa}} \mu^{\text{sc}}_i +  \sum_{i \in I^\complement_{\varepsilon,\kappa}}\mu^{\text{sc}}_i\\
%& \overset{\eqref{i-th-marginal-mu}}{=} \kappa \norm{\mu_{I_{\varepsilon,\kappa}}^{\text{sc}}}_1 + \frac \varepsilon\kappa \sum_{i \in I^\complement}\Big( \sum_{j \in J_{\varepsilon,\kappa}} K_{ij} e^{v^{\text{sc}}_j} + \varepsilon\kappa \sum_{j\in J^\complement_{\varepsilon,\kappa}}K_{ij}\Big)\\
& \overset{}{\geq} \kappa \norm{\mu_{I_{\varepsilon,\kappa}}^{\text{sc}}}_1 + (n -n_b) \Big(\frac{mm_b\varepsilon^2 K_{\min}^2 \min_{j\in J_{\varepsilon, \kappa}} \nu_j}{ (n-n_b)m\kappa \varepsilon^2 K_{\min} + n_b\kappa^2 \max_{i\in I_{\varepsilon,\kappa}}\mu_i }+ (m-m_b)\varepsilon^2K_{\min}\Big),
\end{align*}
which gives the claimed result.
Similarly, we can prove the same statement for $\norm{\nu^{\text{sc}}}_1$.

\subsection{Proof of Proposition~\ref{proposition_error_in_marginals}}

We define the distance function $\varrho: \R_+ \times \R_+ \mapsto [0, \infty]$ by $\varrho(a,b) = b - a + a \log(\frac ab).$
While $\varrho$ is not a metric, it is easy to see that $\varrho$ is not nonnegative and satisfies $\varrho(a,b) =0$ iff $a=b$.
The violations are computed through the following function: 
\begin{equation*}
	d_{\varrho}(\gamma,\beta) = \sum_{i=1}^n \varrho(\gamma_i,\beta_i), \text{ for } \gamma, \beta \in \R^n_+.
\end{equation*}
Note that if $\gamma,\beta$ are two vectors of positive entries, $d_{\varrho}(\gamma,\beta)$ will return some measurement on how far they are from each other. The next Lemma is from~\cite{khalilabid2018} (see Lemma 7 herein).
\begin{lemma}
\label{lem:pinsker}
For any $\gamma, \beta \in \R^n_+$, the following generalized Pinsker inequality holds 
\begin{align*}
\norm{\gamma - \beta}_1 \leq \sqrt{7 (\norm{\gamma}_1\wedge \norm{\beta}_1)d_{\varrho}(\gamma,\beta)}
\end{align*}
\end{lemma}
By~\eqref{i-th-marginal-mu}, we have
\begin{align*}
d_\varrho({\mu} ,{\mu}^{\text{sc}}) &= \sum_{i=1}^n  {\mu}^{\text{sc}}_i - {\mu}_i + {\mu}_i  \log\Big(\frac{{\mu}_i}{{\mu}^{\text{sc}}_i }\Big)\\
&= \sum_{i\in I_{\varepsilon,\kappa}} (\kappa-1)\mu_i - \mu_i\log(\kappa) + \sum_{i\in I^\complement_{\varepsilon,\kappa}}\frac \varepsilon\kappa\sum_{j=1}^m K_{ij} e^{v^{\text{sc}}_j} - \mu_i + \mu_i \log\Big(\frac{\mu_i}{\frac \varepsilon\kappa\sum_{j=1}^m K_{ij} e^{v^{\text{sc}}_j}}\Big)\\
&= \sum_{i\in I_{\varepsilon,\kappa}} (\kappa-\log(\kappa)-1)\mu_i  + \sum_{i\in I^\complement_{\varepsilon,\kappa}}\frac \varepsilon\kappa\sum_{j=1}^m K_{ij} e^{v^{\text{sc}}_j} - \mu_i + \mu_i \log\Big(\frac{\mu_i}{\frac \varepsilon\kappa\sum_{j=1}^m K_{ij} e^{v^{\text{sc}}_j}}\Big).
% &\leq  \sum_{i\in I^\complement_{\varepsilon,\kappa}}\frac \varepsilon\kappa\sum_{j=1}^m K_{ij} e^{v^{\text{sc}}_j} - \mu_i + \mu_i \log\Big(\frac{\mu_i}{\frac \varepsilon\kappa\sum_{j=1}^m K_{ij} e^{v^{\text{sc}}_j}}\Big)
\end{align*}
Now by~\eqref{bound_on_v}, we have in one hand 
\begin{align*}
\sum_{i\in I^\complement_{\varepsilon,\kappa}}\frac \varepsilon\kappa\sum_{j=1}^m K_{ij} e^{v^{\text{sc}}_j}&= \sum_{i\in I^\complement_{\varepsilon,\kappa}}\frac \varepsilon\kappa \Big(\sum_{j\in J_{\varepsilon,\kappa}}K_{ij} e^{v^{\text{sc}}_j} + \varepsilon \kappa\sum_{j\in J^\complement_{\varepsilon,\kappa}}K_{ij}\Big)\\
&\leq \sum_{i\in I^\complement_{\varepsilon,\kappa}}\frac \varepsilon\kappa \Big(m_b \max_{i,j}K_{ij}\frac{\max_{j \in J_{\varepsilon,\kappa}} \nu_j}{n\varepsilon K_{\min}} + (m - m_b)\varepsilon\kappa\max_{i,j}K_{ij}\Big) \\
&\leq (n-n_b)\Big(\frac{m_b\max_{j} \nu_j}{n\kappa K_{\min}} + (m- m_b) \varepsilon^2\Big).
\end{align*}
On the other hand, we get
\begin{align*}
\frac \varepsilon\kappa\sum_{j=1}^m K_{ij} e^{v^{\text{sc}}_j}&=\frac \varepsilon\kappa \Big(\sum_{j\in J_{\varepsilon,\kappa}}K_{ij} e^{v^{\text{sc}}_j} + \varepsilon \kappa\sum_{j\in J^\complement_{\varepsilon,\kappa}}K_{ij}\Big)\\
&\geq m_bK_{\min} \frac{m\varepsilon^2K_{\min}\min_{j \in J_{\varepsilon,\kappa}}\nu_j}{\kappa((n-n_b)m\varepsilon^2K_{\min} + m\varepsilon^2K_{\min} + n_b\kappa\max_{i\in I_{\varepsilon,\kappa}}\mu_i)}\\
&\qquad +\varepsilon^2 (m- m_b) K_{\min}\\
&\geq \frac{mm_b\varepsilon^2(K_{\min})^2\min_{j \in J_{\varepsilon,\kappa}}\nu_j}{\kappa((n-n_b)m\varepsilon^2K_{\min}+ m\varepsilon^2K_{\min} + n_b\kappa\max_{i\in I_{\varepsilon,\kappa}}\mu_i)}\\
&\qquad +\varepsilon^2 (m- m_b) K_{\min}\\
&\geq \frac{mm_b\varepsilon^2K_{\min}^2\min_{j \in J_{\varepsilon,\kappa}}\nu_j}{\kappa((n-n_b)m\varepsilon^2K_{\min}+ m\varepsilon^2K_{\min} + n_b\kappa\max_{i\in I_{\varepsilon,\kappa}}\mu_i)}.
\end{align*}
Then 
\begin{align*}
\frac{1}{\frac \varepsilon\kappa\sum_{j=1}^m K_{ij} e^{v^{\text{sc}}_j}} &\leq \frac{\kappa((n-n_b)m\varepsilon^2K_{\min}+ m\varepsilon^2K_{\min} + n_b\kappa\max_{i\in I_{\varepsilon,\kappa}}\mu_i)}{mm_b\varepsilon^2 K_{\min}^2\min_{j \in J_{\varepsilon,\kappa}}\nu_j}\\
&\leq \frac{\kappa(n-n_b+ 1)}{m_bK_{\min}\min_{j \in J_{\varepsilon,\kappa}}\nu_j} + \frac{n_b\kappa^2\max_{i\in I_{\varepsilon,\kappa}}\mu_i}{mm_b\varepsilon^2K_{\min}^2\min_{j \in J_{\varepsilon,\kappa}}\nu_j}.
\end{align*}
It entails 
\begin{align*}
&\sum_{i\in I^\complement_{\varepsilon,\kappa}}\frac \varepsilon\kappa\sum_{j=1}^m K_{ij} e^{v^{\text{sc}}_j} - \mu_i + \mu_i \log\Big(\frac{\mu_i}{\frac \varepsilon\kappa\sum_{j=1}^m K_{ij} e^{v^{\text{sc}}_j}}\Big)\\
&\leq (n-n_b)\bigg(\frac{m_b}{n\kappa\min_{i,j} K_{ij}} + (m- m_b) \varepsilon^2 - \min_{i}\mu_i\\
&\qquad + \max_{i}\mu_i\log\Big(\frac{\kappa(n-n_b+ 1)\max_{i}\mu_i}{m_bK_{\min}\min_{j \in J_{\varepsilon,\kappa}}\nu_j} + \frac{n_b\kappa^2(\max_{i}\mu_i)^2}{mm_b\varepsilon^2 K_{\min}^2\min_{j \in J_{\varepsilon,\kappa}}\nu_j}\Big)
\bigg).
\end{align*}
Therefore
\begin{align*}
d_\varrho({\mu},{\mu}^{\text{sc}}) &\leq n_b(\kappa-\log(\kappa)-1)\max_{i} \mu_i + (n-n_b)\bigg(\frac{m_b\max_{j}\nu_j}{n\kappa\min_{i,j} K_{ij}} + (m- m_b) \varepsilon^2 - \min_{i}\mu_i\\
&\qquad + \max_{i} \mu_i\log\Big(\frac{\kappa(n-n_b+ 1)\max_{i} \mu_i}{m_bK_{\min}\min_{j \in J_{\varepsilon,\kappa}}\nu_j} + \frac{n_b\kappa^2(\max_{i} \mu_i)^2}{mm_b\varepsilon^2 K_{\min}^2\min_{j \in J_{\varepsilon,\kappa}}\nu_j}\Big).
\end{align*}
Finally, by Lemma~\ref{lem:pinsker} we obtain
\begin{align*}
\norm{{\mu} -{\mu}^{\text{sc}}}^2_1 \leq & n_b(\kappa-\log(\kappa)-1)\max_{i} \mu_i + 7(n-n_b)\bigg(\frac{m_b\max_{j}\nu_j}{n\kappa\min_{i,j} K_{ij}} + (m- m_b) \varepsilon^2 - \min_{i}\mu_i\\
&+ \max_{i} \mu_i\log\Big(\frac{\kappa(n-n_b+ 1)\max_{i} \mu_i}{m_bK_{\min}\min_{j \in J_{\varepsilon,\kappa}}\nu_j} + \frac{n_b\kappa^2(\max_{i} \mu_i)^2}{mm_b\varepsilon^2K_{\min}^2\min_{j \in J_{\varepsilon,\kappa}}\nu_j}\Big).%\bigg\}^{1/2}
\end{align*}
Proof of the upper bound for $\norm{\nu - {\nu}^{\text{sc}}}^2_1$ follows the same lines as above.



\subsection{Bounding dual objective values}

For each vectors $u \in \mathcal{U}_{\text{sc}}$ and $v\in \mathcal{V}_{\text{sc}}$ we define its rearrgement 
\begin{equation*}
\dt{u} =(u_{I_{\varepsilon,\kappa}}^\top, u_{I^\complement_{\varepsilon,\kappa}}^\top)^\top = (u_{I_{\varepsilon,\kappa}}^\top, \log(\frac\varepsilon\kappa)\mathbf1_{n- n_b}^\top)^\top,
\end{equation*}
and 
\begin{equation*}
\dt{v} =(v_{J_{\varepsilon,\kappa}}^\top, v_{J^\complement_{\varepsilon,\kappa}}^\top)^\top = (v_{J_{\varepsilon,\kappa}}^\top, \log(\varepsilon\kappa)\mathbf 1_{m- m_b}^\top)^\top.
\end{equation*}


\begin{proposition}
Let $k\geq 1$ and $u^{sc}_k, v^{sc}_k$ be generated by Screenkhorn. Then, ones has 
\begin{align*}
\Psi_{\varepsilon, \kappa}(u^{\text{sc}}_k ,v^{\text{sc}}_k) -\Psi(u^\star, v^\star)
&\leq \norm{\kappa \dt{u}^{\text{sc}}_k - \dt{u}^\star}_\infty \norm{\tilde{B}(\dt{u}^{\text{sc}}_k ,\dt{v}^{\text{sc}}_k)\mathbf 1_m - \dt{\mu}}_1 + \norm{\kappa^{-1}\dt{v}^{\text{sc}}_k - \dt{v}^\star}_\infty \norm{\tilde{B}(\dt{u}^{\text{sc}}_k ,\dt{v}^{\text{sc}}_k)^\top\mathbf 1_n - \dt{\nu}}_1\\
&\qquad + (1- \kappa)\inr{\dt{u}^{\text{sc}}_k ,\tilde{B}(\dt{u}^{\text{sc}}_k ,\dt{v}^{\text{sc}}_k)\mathbf 1_m} + (1- \kappa^{-1})\inr{\dt{v}^{\text{sc}}_k ,\tilde{B}(\dt{u}^{\text{sc}}_k ,\dt{v}^{\text{sc}}_k)^\top\mathbf 1_n}.
\end{align*}

\begin{align*}
\Psi_{\varepsilon, \kappa}(u^{\text{sc}}_k ,v^{\text{sc}}_k) -\Psi(u^\star, v^\star)
&\leq \norm{\kappa {u}^{\text{sc}}_k - {u}^\star}_\infty \norm{\tilde{B}({u}^{\text{sc}}_k ,{v}^{\text{sc}}_k)\mathbf 1_m - {\mu}}_1 + \norm{\kappa^{-1}{v}^{\text{sc}}_k - {v}^\star}_\infty \norm{\tilde{B}({u}^{\text{sc}}_k ,{v}^{\text{sc}}_k)^\top\mathbf 1_n - \dt{\nu}}_1\\
&\qquad + (1- \kappa)\inr{{u}^{\text{sc}}_k ,\tilde{B}({u}^{\text{sc}}_k ,{v}^{\text{sc}}_k)\mathbf 1_m} + (1- \kappa^{-1})\inr{{v}^{\text{sc}}_k ,\tilde{B}({u}^{\text{sc}}_k ,{v}^{\text{sc}}_k)^\top\mathbf 1_n}.
\end{align*}

\newpage
\begin{align*}
\Psi_{\varepsilon, \kappa}(u^{\text{sc}}_k ,v^{\text{sc}}_k) -\Psi(u^\star, v^\star)
&\leq \norm{\kappa {u}^{\text{sc}}_k - {u}^\star}_\infty \norm{\mu ^{\text{sc}}_k- {\mu}}_1 + \norm{\kappa^{-1}{v}^{\text{sc}}_k - {v}^\star}_\infty \norm{\nu^{\text{sc}}_k - {\nu}}_1\\
&\qquad + (1- \kappa)\inr{{u}^{\text{sc}}_k ,\mu ^{\text{sc}}_k} + (1- \kappa^{-1})\inr{{v}^{\text{sc}}_k ,\nu ^{\text{sc}}_k}.
\end{align*}

\end{proposition}

\newpage
\begin{proof}

Let $k\geq 0$ and $u^{sc}_k, v^{sc}_k$ be generated by Algorithm~\ref{screenkhorn} and $(u^\star, v^\star)$ be a solution of~\ref{sinkhorn-dual}. Then 
\begin{align*}
&\Psi_{\varepsilon, \kappa}(u^{sc}_k,v^{sc}_k) - \Psi(u^\star, v^\star)\\
& = \mathbf1_{n_b}^\top B(u_{I_{\varepsilon,\kappa}}, v_{J_{\varepsilon,\kappa}}) \mathbf 1_{m_b} + \varepsilon \kappa (e^{u_{I_{\varepsilon,\kappa}}})^\top K_{(I_{\varepsilon,\kappa}, J^\complement_{\varepsilon,\kappa})} + \varepsilon \kappa^{-1} K_{(I^\complement_{\varepsilon,\kappa}, J_{\varepsilon,\kappa})}e^{v_{J_{\varepsilon,\kappa}}}\\
&\qquad - \kappa \mu_{I_{\varepsilon,\kappa}}^\top u_{I_{\varepsilon,\kappa}} - \kappa^{-1} \nu_{J_{\varepsilon,\kappa}}^\top v_{J_{\varepsilon,\kappa}} + \Xi
\end{align*}

% \begin{proof}
Let us define $\tilde{K}$ a rearrgement of $K$ with respect to the active sets $I_{\varepsilon,\kappa}$ and $J_{\varepsilon,\kappa}$as follows:
\begin{equation*}
\tilde{K} = 
\begin{bmatrix}
K_{(I_{\varepsilon,\kappa}, J_{\varepsilon,\kappa})} & K_{(I_{\varepsilon,\kappa}, J^\complement_{\varepsilon,\kappa})}\\
K_{(I^\complement_{\varepsilon,\kappa}, J_{\varepsilon,\kappa})} &K_{(I^\complement_{\varepsilon,\kappa}, J^\complement_{\varepsilon,\kappa})}
\end{bmatrix}.
\end{equation*}
For each vectors $u \in \mathcal{U}_{\text{sc}}$ and $v\in \mathcal{V}_{\text{sc}}$ we define its rearrgement 
\begin{equation*}
\dt{u} =(u_{I_{\varepsilon,\kappa}}^\top, u_{I^\complement_{\varepsilon,\kappa}}^\top)^\top = (u_{I_{\varepsilon,\kappa}}^\top, \log(\frac\varepsilon\kappa)\mathbf1_{n- n_b}^\top)^\top,
\end{equation*}
and 
\begin{equation*}
\dt{v} =(v_{J_{\varepsilon,\kappa}}^\top, v_{J^\complement_{\varepsilon,\kappa}}^\top)^\top = (v_{J_{\varepsilon,\kappa}}^\top, \log(\varepsilon\kappa)\mathbf 1_{m- m_b}^\top)^\top.
\end{equation*}
We then have 
\begin{equation*}
\Psi_{\varepsilon, \kappa} (u,v) = \mathbf 1_n^\top \tilde{B}(\dt{u}, \dt{v}) \mathbf 1_m - \kappa \dt{\mu}^\top \dt{u} - \kappa^{-1} \dt{\nu}^\top \dt{v},
\end{equation*}
where
\begin{equation*}
  \tilde{B}(\dt{u}, \dt{v}) = \Delta(e^{\dt{u}})\tilde{K} \Delta(e^{\dt{v}}).
\end{equation*}


Let us fix $k\geq 1$ and consider the convex function
\begin{equation*}
	(\hat u, \hat v) \mapsto \inr{\mathbf 1_n, \tilde{B}(\dt{\hat u}^{\text{}} ,\dt{\hat v}^{\text{}})\mathbf 1_m} - \inr{\kappa\dt{\hat u}, \tilde{B}(\dt{u}^{\text{sc}}_k ,\dt{v}^{\text{sc}}_k)\mathbf 1_m} - \inr{\kappa^{-1}\dt{\hat v}, \tilde{B}(\dt{u}^{\text{sc}}_k ,\dt{v}^{\text{sc}}_k)^\top\mathbf 1_n}.
\end{equation*}
Gradient inequality of any convex function g at point $x_o$ reads as $g(x_o) \geq g(x) + \inr{\nabla g(x), x_o - x}, \text{ for all } x \in \textbf{dom}(g).$
Applying the latter fact to the above function at point ($u^{\star}, v^{\star})$ we obtain
\begin{align*}
&\inr{\mathbf 1_n, \tilde{B}(\dt{u}^{\text{sc}}_k ,\dt{v}^{\text{sc}}_k)\mathbf 1_m} - \inr{\kappa\dt{u}^{\text{sc}}_k, \tilde{B}(\dt{u}^{\text{sc}}_k ,\dt{v}^{\text{sc}}_k)\mathbf 1_m} - \inr{\kappa^{-1}\dt{v}^{\text{sc}}_k, \tilde{B}(\dt{u}^{\text{sc}}_k ,\dt{v}^{\text{sc}}_k)^\top\mathbf 1_n}\\
& - \big(\inr{\mathbf 1_n, \tilde{B}(\dt{u}^{\star} ,\dt{v}^{\star})\mathbf 1_m}  
- \inr{\kappa\dt{u}^{\star{}}, \tilde{B}(\dt{u}^{\text{sc}}_k ,\dt{v}^{\text{sc}}_k)\mathbf 1_m} 
- \inr{\kappa^{-1}\dt{v}^\star, \tilde{B}(\dt{u}^{\text{sc}}_k ,\dt{v}^{\text{sc}}_k)^\top\mathbf 1_n} \big)\\
& \leq \inr{\dt{u}^{\text{sc}}_k - \dt{u}^\star, (1-\kappa) \tilde{B}(\dt{u}^{\text{sc}}_k ,\dt{v}^{\text{sc}}_k)\mathbf 1_m} 
+ \inr{\dt{v}^{\text{sc}}_k - \dt{v}^\star, (1-\kappa^{-1}) \tilde{B}(\dt{u}^{\text{sc}}_k ,\dt{v}^{\text{sc}}_k)^\top\mathbf 1_n}.
\end{align*}
Moreover,
\begin{align*}
\Psi_{\varepsilon, \kappa} (u^{\text{sc}}_k ,v^{\text{sc}}_k) -\Psi(u^\star, v^\star)
&= \inr{\mathbf 1_n, \tilde{B}(\dt{u}^{\text{sc}}_k ,\dt{v}^{\text{sc}}_k)\mathbf 1_m} - \inr{\kappa\dt{u}^{\text{sc}}_k, \tilde{B}(\dt{u}^{\text{sc}}_k ,\dt{v}^{\text{sc}}_k)\mathbf 1_m} - \inr{\kappa^{-1}\dt{v}_k^{\text{sc}}, \tilde{B}(\dt{u}_k^{\text{sc}}, \dt{v}^{\text{sc}}_k)\mathbf 1_n^\top}\\
& - \big( 
\inr{\mathbf 1_n, \tilde{B}(\dt{u}^{\star} ,\dt{v}^{\star})\mathbf 1_m}  
- \inr{\dt{u}^{\star{}}, \tilde{B}(\dt{u}^{\text{sc}}_k ,\dt{v}^{\text{sc}}_k)\mathbf 1_m} 
- \inr{\dt{v}^\star, \tilde{B}(\dt{u}^{\text{sc}}_k ,\dt{v}^{\text{sc}}_k)^\top\mathbf 1_n} \big)\\
&+ \inr{\kappa \dt{u}^{\text{sc}}_k - \dt{u}^{\star}, \tilde{B}(\dt{u}^{\text{sc}}_k ,\dt{v}^{\text{sc}}_k)\mathbf 1_m - \dt{\mu}} + \inr{\kappa^{-1}\dt{v}^{\text{sc}}_k - \dt{v}^{\star}, \tilde{B}(\dt{u}^{\text{sc}}_k ,\dt{v}^{\text{sc}}_k)^\top\mathbf 1_n - \dt{\nu}}.
\end{align*}
Hence,
\begin{align*}
&\Psi_{\varepsilon, \kappa} (u^{\text{sc}}_k ,v^{\text{sc}}_k) -\Psi(u^\star, v^\star)\\
&\qquad + \big( \inr{\mathbf 1_n, \tilde{B}(\dt{u}^{\star} ,\dt{v}^{\star})\mathbf 1_m}  
- \inr{\dt{u}^{\star{}}, \tilde{B}(\dt{u}^{\text{sc}}_k ,\dt{v}^{\text{sc}}_k)\mathbf 1_m} 
- \inr{\dt{v}^\star, \tilde{B}(\dt{u}^{\text{sc}}_k ,\dt{v}^{\text{sc}}_k)^\top\mathbf 1_n} \big)\\
& \qquad -\inr{\kappa \dt{u}^{\text{sc}}_k - \dt{u}^{\star}, \tilde{B}(\dt{u}^{\text{sc}}_k ,\dt{v}^{\text{sc}}_k)\mathbf 1_m - \dt{\mu}} -\inr{\kappa^{-1}\dt{v}^{\text{sc}}_k - \dt{v}^{\star}, \tilde{B}(\dt{u}^{\text{sc}}_k ,\dt{v}^{\text{sc}}_k)^\top\mathbf 1_n - \dt{\nu}}\\
& \leq \inr{\dt{u}^{\text{sc}}_k - \dt{u}^\star, (1-\kappa) \tilde{B}(\dt{u}^{\text{sc}}_k ,\dt{v}^{\text{sc}}_k)\mathbf 1_m} 
+ \inr{\dt{v}^{\text{sc}}_k - \dt{v}^\star, (1-\kappa^{-1}) \tilde{B}(\dt{u}^{\text{sc}}_k ,\dt{v}^{\text{sc}}_k)^\top\mathbf 1_n}\\
&\qquad + \big(\inr{\mathbf 1_n, \tilde{B}(\dt{u}^{\star} ,\dt{v}^{\star})\mathbf 1_m}  
- \inr{\kappa\dt{u}^{\star{}}, \tilde{B}(\dt{u}^{\text{sc}}_k ,\dt{v}^{\text{sc}}_k)\mathbf 1_m} 
- \inr{\kappa^{-1}\dt{v}^\star, \tilde{B}(\dt{u}^{\text{sc}}_k ,\dt{v}^{\text{sc}}_k)^\top\mathbf 1_n} \big)
\end{align*}
Then,
\begin{align*}
&\Psi_{\varepsilon, \kappa} (u^{\text{sc}}_k ,v^{\text{sc}}_k) -\Psi(u^\star, v^\star)\\
&\leq \inr{\dt{u}^{\text{sc}}_k - \dt{u}^\star, (1-\kappa) \tilde{B}(\dt{u}^{\text{sc}}_k ,\dt{v}^{\text{sc}}_k)\mathbf 1_m} 
+ \inr{\dt{v}^{\text{sc}}_k - \dt{v}^\star, (1-\kappa^{-1}) \tilde{B}(\dt{u}^{\text{sc}}_k ,\dt{v}^{\text{sc}}_k)^\top\mathbf 1_n}\\
&\qquad + \big(\inr{\mathbf 1_n, \tilde{B}(\dt{u}^{\star} ,\dt{v}^{\star})\mathbf 1_m}  
- \inr{\kappa\dt{u}^{\star{}}, \tilde{B}(\dt{u}^{\text{sc}}_k ,\dt{v}^{\text{sc}}_k)\mathbf 1_m} 
- \inr{\kappa^{-1}\dt{v}^\star, \tilde{B}(\dt{u}^{\text{sc}}_k ,\dt{v}^{\text{sc}}_k)^\top\mathbf 1_n} \big)
\\
&\qquad + \inr{\kappa \dt{u}^{\text{sc}}_k - \dt{u}^{\star}, \tilde{B}(\dt{u}^{\text{sc}}_k ,\dt{v}^{\text{sc}}_k)\mathbf 1_m - \dt{\mu}} + \inr{\kappa^{-1}\dt{v}^{\text{sc}}_k - \dt{v}^{\star}, \tilde{B}(\dt{u}^{\text{sc}}_k ,\dt{v}^{\text{sc}}_k)^\top\mathbf 1_n - \dt{\nu}}\\
&\qquad - \big(\inr{\mathbf 1_n, \tilde{B}(\dt{u}^{\star} ,\dt{v}^{\star})\mathbf 1_m}  
- \inr{\dt{u}^{\star{}}, \tilde{B}(\dt{u}^{\text{sc}}_k ,\dt{v}^{\text{sc}}_k)\mathbf 1_m} 
- \inr{\dt{v}^\star, \tilde{B}(\dt{u}^{\text{sc}}_k ,\dt{v}^{\text{sc}}_k)^\top \mathbf 1_n} \big).
\end{align*}
So
\begin{align*}
&\Psi_{\varepsilon, \kappa} (u^{\text{sc}}_k ,v^{\text{sc}}_k) -\Psi(u^\star, v^\star)\\
&\leq  \inr{\kappa \dt{u}^{\text{sc}}_k - \dt{u}^{\star}, \tilde{B}(\dt{u}^{\text{sc}}_k ,\dt{v}^{\text{sc}}_k)\mathbf 1_m - \dt{\mu}} +\inr{\kappa^{-1}\dt{v}^{\text{sc}}_k - \dt{v}^{\star}, \tilde{B}(\dt{u}^{\text{sc}}_k ,\dt{v}^{\text{sc}}_k)^\top\mathbf 1_n - \dt{\nu}}\\
&\qquad + (1- \kappa)\inr{\dt{u}^{\text{sc}}_k ,\tilde{B}(\dt{u}^{\text{sc}}_k ,\dt{v}^{\text{sc}}_k)\mathbf 1_m} + (1- \kappa^{-1})\inr{\dt{v}^{\text{sc}}_k ,\tilde{B}(\dt{u}^{\text{sc}}_k ,\dt{v}^{\text{sc}}_k)^\top\mathbf 1_n}.
\end{align*}
Applying Holder's inequality entails
\begin{align*}
&\Psi_{\varepsilon, \kappa}(u^{\text{sc}}_k ,v^{\text{sc}}_k) -\Psi(u^\star, v^\star)\\
&\leq \norm{\kappa \dt{u}^{\text{sc}}_k - \dt{u}^\star}_\infty \norm{\tilde{B}(\dt{u}^{\text{sc}}_k ,\dt{v}^{\text{sc}}_k)\mathbf 1_m - \dt{\mu}}_1 + \norm{\kappa^{-1}\dt{v}^{\text{sc}}_k - \dt{v}^\star}_\infty \norm{\tilde{B}(\dt{u}^{\text{sc}}_k ,\dt{v}^{\text{sc}}_k)^\top\mathbf 1_n - \dt{\nu}}_1\\
&\qquad + (1- \kappa)\inr{\dt{u}^{\text{sc}}_k ,\tilde{B}(\dt{u}^{\text{sc}}_k ,\dt{v}^{\text{sc}}_k)\mathbf 1_m} + (1- \kappa^{-1})\inr{\dt{v}^{\text{sc}}_k ,\tilde{B}(\dt{u}^{\text{sc}}_k ,\dt{v}^{\text{sc}}_k)^\top\mathbf 1_n}.
\end{align*}
Next, we bound the right-hand side of this inequality.

If $r \in I^\complement_{\varepsilon, \kappa}$ then 
\begin{align*}
|(\dt{u}^{\text{sc}}_k)_{r} - \dt{u}^\star_r| &= \bigg| \log(\frac \varepsilon\kappa) - \log(\frac{\mu_r}{\sum_{j=1}^m K_{rj}e^{v^\star_j}})\bigg|\\
&= \bigg|\log\Big(\frac{\frac \varepsilon\kappa}{\frac{\mu_r}{\sum_{j=1}^m K_{rj}e^{v^\star_j}}}\Big) \bigg|\\
&= \bigg|\log\bigg(\frac{\sum_{j=1}^m K_{rj}e^{v^\star_j}}{\sum_{j=1}^m \frac{\kappa \mu_r}{m\varepsilon}}\bigg) \bigg|\\
& \leq \bigg|\log\bigg(\max_{1\leq j \leq m} \frac{K_{rj}e^{v^\star_j}}{\frac{\kappa \mu_r}{m\varepsilon}}\bigg) \bigg|\\
& \leq \big|\max_{1\leq j \leq m} (v_j^\star - \log(\frac{\kappa \mu_r}{m\varepsilon})\big|\\
& \leq \norm{v^\star - \log(\frac{\kappa \mu_r}{m\varepsilon})}_\infty \\
& \leq \norm{v^\star - v^{\text{sc}}_k - \log(\frac{\mu_r}{m\varepsilon^2})}_\infty \\
& \leq \norm{v^\star - v^{\text{sc}}_k}_\infty +  \log(m\varepsilon^2).
\end{align*}

If $r \in I_{\varepsilon, \kappa}$ then 
\begin{align*}
|(\dt{u}^{\text{sc}}_k)_{r} - \dt{u}^\star_r| &= \bigg|\log\bigg(\frac{\kappa\mu_r}{\sum_{j\in J_{\varepsilon,\kappa}}K_{rj}e^{(v^{\text{sc}}_k)_j} + \varepsilon\kappa \sum_{j\in J^\complement_{\varepsilon,\kappa}}K_{rj}}\bigg) - \log\bigg(\frac{\mu_r}{\sum_{j=1}^m K_{rj}e^{v^\star_j}}\bigg)\bigg|\\
&= \bigg|\log\bigg(\frac{\kappa\mu_r}{\sum_{j=1}^m K_{rj}e^{(v^{\text{sc}}_k)_j}}\bigg) - \log\bigg(\frac{\mu_r}{\sum_{j=1}^m K_{rj}e^{v^\star_j}}\bigg)\bigg|\\
&= \bigg|\log\bigg(\frac{\kappa\sum_{j=1}^m K_{rj}e^{v^\star_j}}{\sum_{j=1}^m K_{rj}e^{(v^{\text{sc}}_k)_j}}\bigg)\bigg|\\
& \leq \bigg|\log\bigg(\frac{\kappa\sum_{j=1}^m K_{rj}e^{v^\star_j}}{\sum_{j=1}^m K_{rj}e^{(v^{\text{sc}}_k)_j}}\bigg)\bigg|\\
& \leq \norm{(v^{\text{sc}}_k) - v^\star - \log(\kappa)}_\infty\\
& \leq \norm{(v^{\text{sc}}_k) - v^\star}_\infty (\kappa \leq 1)
\end{align*}

If $s \in J^\complement_{\varepsilon, \kappa}$ then 
\begin{align*}
|(\dt{v}^{\text{sc}}_k)_{s} - \dt{v}^\star_s| &= \bigg| \log(\varepsilon\kappa) - \log(\frac{\nu_s}{\sum_{i=1}^n K_{is}e^{u^\star_i}})\bigg|\\
&= \bigg|\log\Big(\frac{\varepsilon\kappa}{\frac{\nu_s}{\sum_{i=1}^n K_{is}e^{u^\star_i}}}\Big) \bigg|\\
&= \bigg|\log\bigg(\frac{\sum_{j=1}^n K_{is}e^{u^\star_i}}{\sum_{i=1}^n \frac{\nu_s}{n\kappa\varepsilon}}\bigg) \bigg|\\
& \leq \bigg|\log\bigg(\max_{1\leq i \leq n} \frac{K_{is}e^{u^\star_i}}{\frac{\nu_s}{n\kappa\varepsilon}}\bigg) \bigg|\\
& \leq \big|\max_{1\leq i \leq n} (u_i^\star - \log(\frac{\nu_s}{n\kappa\varepsilon})\big|\\
& \leq \norm{u^\star - \log(\frac{\kappa \mu_r}{m\varepsilon})}_\infty \\
& \leq \norm{u^\star - u^{\text{sc}}_k - \log(\frac{\nu_s}{n\varepsilon^2})}_\infty \\
& \leq \norm{u^\star - u^{\text{sc}}_k}_\infty +  \log(n\varepsilon^2).
\end{align*}

If $r \in I_{\varepsilon, \kappa}$ then 
\begin{align*}
|(\dt{u}^{\text{sc}}_k)_{r} - \dt{u}^\star_r| &= \bigg|\log\bigg(\frac{\kappa\mu_r}{\sum_{j\in J_{\varepsilon,\kappa}}K_{rj}e^{(v^{\text{sc}}_k)_j} + \varepsilon\kappa \sum_{j\in J^\complement_{\varepsilon,\kappa}}K_{rj}}\bigg) - \log\bigg(\frac{\mu_r}{\sum_{j=1}^m K_{rj}e^{v^\star_j}}\bigg)\bigg|\\
&= \bigg|\log\bigg(\frac{\kappa\mu_r}{\sum_{j=1}^m K_{rj}e^{(v^{\text{sc}}_k)_j}}\bigg) - \log\bigg(\frac{\mu_r}{\sum_{j=1}^m K_{rj}e^{v^\star_j}}\bigg)\bigg|\\
&= \bigg|\log\bigg(\frac{\kappa\sum_{j=1}^m K_{rj}e^{v^\star_j}}{\sum_{j=1}^m K_{rj}e^{(v^{\text{sc}}_k)_j}}\bigg)\bigg|\\
& \leq \bigg|\log\bigg(\frac{\kappa\sum_{j=1}^m K_{rj}e^{v^\star_j}}{\sum_{j=1}^m K_{rj}e^{(v^{\text{sc}}_k)_j}}\bigg)\bigg|\\
& \leq \norm{(v^{\text{sc}}_k) - v^\star - \log(\kappa)}_\infty\\
& \leq \norm{(v^{\text{sc}}_k) - v^\star}_\infty (\kappa \leq 1)
\end{align*}


\newpage
\begin{align*} 
\nabla_u \Psi_{\varepsilon, \kappa}(u^{\text{sc}},v^{\text{sc}}) &= e^{u^{\text{sc}}_{I_{\varepsilon,\kappa}}} \odot K_{(I_{\varepsilon,\kappa}, J_{\varepsilon,\kappa})} e^{v^{\text{sc}}_{J_{\varepsilon,\kappa}}} + \varepsilon\kappa e^{u^{\text{sc}}_{I_{\varepsilon,\kappa}}} \odot K_{(I_{\varepsilon,\kappa}, J^\complement_{\varepsilon,\kappa})} \mathbf 1_{m_b} - \kappa \mu_{I_{\varepsilon,\kappa}}\\
&= 
\begin{bmatrix}
e^{u^{\text{sc}}_{I_{\varepsilon,\kappa}}}\\
e^{u^{\text{sc}}_{I^\complement_{\varepsilon,\kappa}}}
\end{bmatrix}
\tilde{K}^1_O
\begin{bmatrix}
e^{v^{\text{sc}}_{J_{\varepsilon,\kappa}}}\\
e^{v^{\text{sc}}_{J^\complement_{\varepsilon,\kappa}}}
\end{bmatrix}
- \kappa \mu_{I_{\varepsilon,\kappa}},\\
\end{align*}
where 
\begin{equation*}
\tilde{K}^1_O = 
\begin{bmatrix}
K_{(I_{\varepsilon,\kappa}, J_{\varepsilon,\kappa})} & K_{(I_{\varepsilon,\kappa}, J^\complement_{\varepsilon,\kappa})}\\
\mathbf O_{(I^\complement_{\varepsilon,\kappa}, J_{\varepsilon,\kappa})} &\mathbf O_{(I^\complement_{\varepsilon,\kappa}, J^\complement_{\varepsilon,\kappa})}
\end{bmatrix}.
\end{equation*}
and 
\begin{align*} 
\nabla_v \Psi_{\varepsilon, \kappa}(u^{\text{sc}},v^{\text{sc}}) &= e^{v^{\text{sc}}_{J_{\varepsilon,\kappa}}} \odot K^\top_{(I_{\varepsilon,\kappa}, J_{\varepsilon,\kappa})} e^{u^{\text{sc}}_{I_{\varepsilon,\kappa}}} + \frac\varepsilon\kappa e^{v^{\text{sc}}_{J_{\varepsilon,\kappa}}} \odot K^\top_{(I^\complement_{\varepsilon,\kappa}, J_{\varepsilon,\kappa})} \mathbf 1_{n_b} - \kappa^{-1} \nu_{J_{\varepsilon,\kappa}}\\
&= 
\begin{bmatrix}
e^{v^{\text{sc}}_{J_{\varepsilon, \kappa}}}\\
e^{v^{\text{sc}}_{J^\complement_{\varepsilon, \kappa}}}
\end{bmatrix}
\tilde{K}^2_O
\begin{bmatrix}
e^{u^{\text{sc}}_{I_{\varepsilon,\kappa}}}\\
e^{u^{\text{sc}}_{I^\complement_{\varepsilon,\kappa}}}
\end{bmatrix}
- \kappa^{-1} \nu_{J_{\varepsilon,\kappa}}\\
\end{align*}
where 
\begin{equation*}
\tilde{K}^2_O = 
\begin{bmatrix}
K_{(I_{\varepsilon,\kappa}, J_{\varepsilon,\kappa})} &\mathbf O_{(I_{\varepsilon,\kappa}, J^\complement_{\varepsilon,\kappa})}\\
K_{(I^\complement_{\varepsilon,\kappa}, J_{\varepsilon,\kappa})} &\mathbf O_{(I^\complement_{\varepsilon,\kappa}, J^\complement_{\varepsilon,\kappa})}
\end{bmatrix}.
\end{equation*}
Therefore, we obtain 
\begin{equation*}
\nabla_u \Psi_{\varepsilon, \kappa}(u^{\text{sc}},v^{\text{sc}}) = \tilde{B}^1_O(\dt{u}^{\text{sc}},\dt{v}^{\text{sc}}) \mathbf 1_m - \kappa \mu_{I_{\varepsilon,\kappa}} \text{ and } \nabla_v \Psi_{\varepsilon, \kappa}(u^{\text{sc}},v^{\text{sc}}) = \tilde{B}^2_O(\dt{u}^{\text{sc}},\dt{v}^{\text{sc}})^\top \mathbf 1_n- \kappa^{-1} \nu_{J_{\varepsilon,\kappa}}
\end{equation*}

Analogously, one can define 
\begin{align*}
\Psi(u^\star, v^\star) = \mathbf{1}_n \tilde{B}(\dt{u}^\star, \dt{v}^\star) \mathbf{1}_m - \dt{\mu}^\top \dt{u}^\star - \dt{\nu}^\top \dt{v}^\star
\end{align*}
Hence,
\begin{align*} 
 \Psi_{\varepsilon, \kappa} (u^{\text{}} ,v^{\text{}}) -\Psi(u^\star, v^\star)
& = \inr{\mathbf 1_n, \tilde{B}(\dt{u}^{\text{}} ,\dt{v}^{\text{}})\mathbf 1_m} - \inr{\mathbf{1}_n, \tilde{B}(\dt{u}^{\star} ,\dt{v}^{\star})\mathbf 1_m} + \inr{\dt{u}^{\star} - \kappa \dt{u}^{\text{}}, \dt{\mu}} + \inr{\dt{v}^{\star} - \kappa^{-1}\dt{v}^{\text{}}, \dt{\nu}}.
\end{align*}
\end{proof}

\subsection{Additional experimental results}

\end{document}